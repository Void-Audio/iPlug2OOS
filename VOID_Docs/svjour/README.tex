% !TeX spellcheck = en-US
% !TeX encoding = utf8
% !TeX program = pdflatex
% !BIB program = bibtex
% -*- coding:utf-8 mod:LaTeX -*-

% Fix to get cleveref working
% Source: https://tex.stackexchange.com/a/243866/9075
\makeatletter
\def\cl@chapter{}
\makeatother

%\documentclass[natbib,english]{svjour3}                 % onecolumn (standard format)
%\documentclass[smallcondensed,natbib,english]{svjour3}  % onecolumn (ditto)
%\documentclass[smallextended,natbib,english]{svjour3    % onecolumn (second format)
\documentclass[twocolumn,natbib]{svjour3}        % twocolumn
\usepackage[english]{babel}

% Springer requires a journal name
\companyname{VOID Audio}

% Fix for microtype
% Source: https://tex.stackexchange.com/a/24875/9075
\makeatletter
\if@twocolumn
  \renewcommand\normalsize{%
    \@setfontsize\normalsize\@xpt{12.5pt}%
    \abovedisplayskip=3 mm plus6pt minus 4pt
    \belowdisplayskip=3 mm plus6pt minus 4pt
    \abovedisplayshortskip=0.0 mm plus6pt
    \belowdisplayshortskip=2 mm plus4pt minus 4pt
    \let\@listi\@listI}%

  \renewcommand\small{%
    \@setfontsize\small{8.5pt}\@xpt
    \abovedisplayskip 8.5\p@ \@plus3\p@ \@minus4\p@
    \abovedisplayshortskip \z@ \@plus2\p@
    \belowdisplayshortskip 4\p@ \@plus2\p@ \@minus2\p@
    \def\@listi{\leftmargin\leftmargini
      \parsep 0\p@ \@plus1\p@ \@minus\p@
      \topsep 4\p@ \@plus2\p@ \@minus4\p@
      \itemsep0\p@}%
    \belowdisplayskip \abovedisplayskip}
\else
  \if@smallext
    \renewcommand\normalsize{%
      \@setfontsize\normalsize\@xpt\@xiipt
      \abovedisplayskip=3 mm plus6pt minus 4pt
      \belowdisplayskip=3 mm plus6pt minus 4pt
      \abovedisplayshortskip=0.0 mm plus6pt
      \belowdisplayshortskip=2 mm plus4pt minus 4pt
      \let\@listi\@listI}%

    \renewcommand\small{%
      \@setfontsize\small\@viiipt{9.5pt}%
      \abovedisplayskip 8.5\p@ \@plus3\p@ \@minus4\p@
      \abovedisplayshortskip \z@ \@plus2\p@
      \belowdisplayshortskip 4\p@ \@plus2\p@ \@minus2\p@
      \def\@listi{\leftmargin\leftmargini
        \parsep 0\p@ \@plus1\p@ \@minus\p@
        \topsep 4\p@ \@plus2\p@ \@minus4\p@
        \itemsep0\p@}%
      \belowdisplayskip \abovedisplayskip}
  \else
    \renewcommand\normalsize{%
      \@setfontsize\normalsize{9.5pt}{11.5pt}%
      \abovedisplayskip=3 mm plus6pt minus 4pt
      \belowdisplayskip=3 mm plus6pt minus 4pt
      \abovedisplayshortskip=0.0 mm plus6pt
      \belowdisplayshortskip=2 mm plus4pt minus 4pt
      \let\@listi\@listI}%  

    \renewcommand\small{%
      \@setfontsize\small\@viiipt{9.25pt}%
      \abovedisplayskip 8.5\p@ \@plus3\p@ \@minus4\p@
      \abovedisplayshortskip \z@ \@plus2\p@
      \belowdisplayshortskip 4\p@ \@plus2\p@ \@minus2\p@
      \def\@listi{\leftmargin\leftmargini
        \parsep 0\p@ \@plus1\p@ \@minus\p@
        \topsep 4\p@ \@plus2\p@ \@minus4\p@
        \itemsep0\p@}%
      \belowdisplayskip \abovedisplayskip}
  \fi
\fi
\let\footnotesize\small
\makeatother

% cmap has to be loaded before any font package (such as cfr-lm)
\usepackage{cmap}

%enable margin kerning
\RequirePackage[%
  shrink=40,
  final,%
  expansion=alltext,%
  protrusion=alltext-nott]{microtype}%
\DisableLigatures{encoding = T1, family = tt* }

% Set English as language and allow to write hyphenated"=words
\addto\extrasenglish{\languageshorthands{ngerman}\useshorthands{"}}

\usepackage{ifluatex}
\ifluatex
  \usepackage{fontspec}
  \usepackage[english]{selnolig}
\fi

\ifluatex
  \setmainfont{Latin Modern Roman}
  \setsansfont{Latin Modern Sans}
  \setmonofont{Latin Modern Mono}
\else
  \usepackage[%
    rm={oldstyle=false,proportional=true},%
    sf={oldstyle=false,proportional=true},%
    tt={oldstyle=false,proportional=true,variable=false},%
    qt=false%
  ]{cfr-lm}
\fi

\ifluatex
\else
  \usepackage[T1]{fontenc}
  \usepackage[utf8]{inputenc}
\fi

\smartqed  % flush right qed marks

\usepackage{graphicx}
\usepackage{upquote}
\usepackage{booktabs}
\usepackage{paralist}
\usepackage{csquotes}
\usepackage{textcmds}
\usepackage{url}
\makeatletter
\g@addto@macro{\UrlBreaks}{\UrlOrds}
\makeatother

% siunitx updated config (fixes group-four-digits error)
\usepackage[mode=text,group-digits=integer,group-separator={,}]{siunitx}
\sisetup{locale=US}
\newcommand{\numt}[1]{\num{#1}}

\usepackage{xcolor}
\usepackage{listings}
\lstset{%
  basicstyle=\ttfamily,%
  columns=fixed,%
  basewidth=.5em,%
  xleftmargin=0.5cm,%
  captionpos=b}%
\renewcommand{\lstlistingname}{List.}
\usepackage{chngcntr}
\AtBeginDocument{\counterwithout{lstlisting}{section}}

\usepackage{pdfcomment}
\newcommand{\commentontext}[2]{\colorbox{yellow!60}{#1}\pdfcomment[color={0.234 0.867 0.211},hoffset=-6pt,voffset=10pt,opacity=0.5]{#2}}
\newcommand{\commentatside}[1]{\pdfcomment[color={0.045 0.278 0.643},icon=Note]{#1}}
\newcommand{\todo}[1]{\commentatside{#1}}
\newcommand{\TODO}[1]{\commentatside{#1}}

\newcommand*{\doi}[1]{\href{https://doi.org/\detokenize{#1}}{DOI: \detokenize{#1}}}

\usepackage{stfloats}
\fnbelowfloat

\usepackage[all]{hypcap}

\usepackage[acronym,indexonlyfirst,nomain]{glossaries}
\glsdisablehyper
\newacronym{rdbs}{RDB}{Relational Database}
\newacronym{rdbms}{RDBMS}{Relational Database Management System}


\usepackage[capitalise,nameinlink]{cleveref}
\crefname{section}{Sect.}{Sect.}
\Crefname{section}{Section}{Sections}
\crefname{listing}{\lstlistingname}{\lstlistingname}
\Crefname{listing}{Listing}{Listings}

\newcommand{\Vlabel}[1]{\label[line]{#1}\hypertarget{#1}{}}
\newcommand{\lref}[1]{\hyperlink{#1}{\FancyVerbLineautorefname~\ref*{#1}}}

\usepackage{xspace}
\newcommand{\eg}{e.\,g.,\ }
\newcommand{\ie}{i.\,e.,\ }

\DeclareFontFamily{U}{MnSymbolC}{}
\DeclareSymbolFont{MnSyC}{U}{MnSymbolC}{m}{n}
\DeclareFontShape{U}{MnSymbolC}{m}{n}{
  <-6>    MnSymbolC5
  <6-7>   MnSymbolC6
  <7-8>   MnSymbolC7
  <8-9>   MnSymbolC8
  <9-10>  MnSymbolC9
  <10-12> MnSymbolC10
  <12->   MnSymbolC12%
}{}
\DeclareMathSymbol{\powerset}{\mathord}{MnSyC}{180}

\renewcommand{\eg}{e.\,g.,\xspace}
\newcommand{\egc}{e.\,g.,\xspace}

\hyphenation{op-tical net-works semi-conduc-tor data-bases Berm-bach}

\hypersetup{hidelinks,
  colorlinks=true,
  allcolors=black,
  pdfstartview=Fit,
  breaklinks=true}

\companyname{VOID AUDIO}
\versionnumber{0.1.0.0}

\usepackage[math]{blindtext}
\usepackage{mwe}
\usepackage{graphicx} 
\usepackage{amsmath}  
\usepackage{amssymb}   




%%%%%%%%%%%%%%%%%%%%%%%%%%%%%%%%%%%%%%%%%%%%%%%%%%%%%%%%%%%%%%%%%%%%%%%%%%%
%%%%%%%%%%%%%%%%%%%%%%%%%%%%%%%% README %%%%%%%%%%%%%%%%%%%%%%%%%%%%%%%%%%%
%%%%%%%%%%%%%%%%%%%%%%%%%%%%%%%%%%%%%%%%%%%%%%%%%%%%%%%%%%%%%%%%%%%%%%%%%%%
\begin{document}

\title{iPlug2OOS VOID Audio Fork}

\author{Derek Wingard}

\institute{
  D.~Wingard\at
  VOID Audio \\
  \email{fromthevoid.audio@gmail.com}
}

\date{Version Date: 9/10/2025}

\maketitle

\begin{abstract}
    Added quality of life features that make the build process a bit more foolproof.
  \keywords{Foolproof}
\end{abstract}

\section{Version \texttt{v.0.1.0.0} Update List}
\subsection{Key Features Added}
Brief list of all the added functionality

\subsubsection{\texttt{init.sh}}
This script is for setting up the project once the Codespace is started. Edit the configuration file \texttt{config.txt} to specify the project:
\begin{verbatim}
PROJECT_NAME = MyPlugin

MANUFACTURER_NAME = MyCompany
\end{verbatim}
with more specification to the build expected to come in later versions of this script.
Then, activate the init script with:
\begin{lstlisting}
./init.sh
\end{lstlisting}

\subsubsection{\texttt{setup.sh}} 
Rebuilt the \texttt{setup.sh} script that used to exist in the repo. It now gets called from \texttt{init.sh} and is fed arguments from \texttt{config.txt}.

\subsubsection{\protect\texttt{undo\_setup.sh}}
Makes the process a little more foolproof. You can now undo the \texttt{init.sh} script and reset the project folders/files to \texttt{TemplateProject}. Reads correct \texttt{PROJECT\_NAME} from \texttt{config.txt}. The folders \texttt{.github/workflows} and \texttt{.vscode} are \texttt{git} restored so make sure you have the initial state you want to revert to as your previous push or else you will have to manually delete and copy back in the template files. If you have your tracking set up, then simply reset the project with:
\begin{lstlisting}
./undo_setup.sh
\end{lstlisting}
\subsubsection{Updated \texttt{duplicate.py}}
The \texttt{duplicate.py} script was changing my custom scripts so I updated it to no longer touch them.

\subsubsection{Added LaTeX functionality and Docs to the Codespace}
Now includes scripts to setup LaTeX to compile the README.tex file. Edits can be made at \texttt{VOID\_Docs/svjour/README.tex}. When you are ready to compile them, first use:
\begin{lstlisting}
./VOID_Docs/setup_latex.sh
\end{lstlisting}
then you will be able to compile with
\begin{lstlisting}
./VOID_Docs/tex.sh
\end{lstlisting}
Compiles \texttt{README.pdf} to \texttt{VOID\_Docs}.
\subsubsection{Optimized \texttt{Git} Commands}
I added my personal \texttt{git} commands for quality of life stuff. Updated \texttt{devcontainer.json} so they are available instantly when the workspace is built.

\vspace{1em}

\texttt{gitster}:
\begin{verbatim}
------------------------------
read -p "Commit message: " msg
git add .
git commit -m "$msg"
git push origin master
------------------------------
\end{verbatim}

\vspace{1em}

\texttt{gitmain}:
\begin{verbatim}
------------------------------
read -p "Commit message: " msg
git add .
git commit -m "$msg"
git push origin main
------------------------------
\end{verbatim}

\vspace{1em}

\texttt{gitbranch}:
\begin{verbatim}
------------------------------
read -p "Branch name: " branch
read -p "Commit message: " msg
git add .
git commit -m "$msg"
git push origin "$branch"
------------------------------
\end{verbatim}
\section{Expected Next Steps}
As it stands, learning how to use this framework as effeciently as possible will involve improvements to this init build script. Ideally, we will have a set of specifications available in \texttt{config.txt} that can get started on specific build types that require unique logic, for example:
\begin{verbatim}
REQUIRES_REALTIME_DISPLAY = bool
IS_INSTRUMENT = bool
NUM_PARAMS = int
BUILD_AAX = bool
LICENSING_TYPE = string
EXTERNAL_LIBS = string
etc...
\end{verbatim}

\bibliographystyle{spbasic}
\bibliography{paper}

\phantom{.}

Nothing at the moment.
\end{document}
%%%%%%%%%%%%%%%%%%%%%%%%%%%%%%%%%%%%%%%%%%%%%%%%%%%%%%%%%%%%%%%%%%%%%%%%%%%
%%%%%%%%%%%%%%%%%%%%%%%%%%%%%%%% README %%%%%%%%%%%%%%%%%%%%%%%%%%%%%%%%%%%
%%%%%%%%%%%%%%%%%%%%%%%%%%%%%%%%%%%%%%%%%%%%%%%%%%%%%%%%%%%%%%%%%%%%%%%%%%%
